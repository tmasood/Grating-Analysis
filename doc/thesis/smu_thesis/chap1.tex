
\chapter{INTRODUCTION} \label{ch:introduction}

Grating-outcoupled surface emitting (GSE) lasers have operating
characteristics that include narrow beam divergence, single wavelength
emission \cite{Masood04}, high output power \cite{Evans91}, low
voltage and stable polarization. The combination of in-plane light
propagation and surface normal emission make GSE lasers ideal for
electronic and photonic integrated circuits \cite{Eriksson95}. Surface
emission allows complete wafer level processing and testing,
advantages that permitted significant reductions in cost and improved
reliability for integrated circuits and short-wavelength
vertical-cavity surface-emitting lasers (VCSEL) \cite{Evans93}.

The design, fabrication and characterization of AlGaInAs/InP GSE
lasers emitting at wavelengths near 1310 nm is reported. This GSE
architecture is wavelength agnostic and can be applied to any existing
material system used to make semiconductor lasers.

GSE lasers (Fig. \ref{phtd3Dview}) consist of a ridge waveguide
gain section of length $L_{\rm {Ridge}}$, distributed Bragg reflectors
(DBRs) of length $L_{\rm {DBR}}$ at each end of the lasing cavity and an
outcoupler of length $L_{\rm {OC}}$ placed between one of the DBR reflectors
and the active ridge. The shallow DBRs provide wavelength selective
feedback for single-frequency operation.

\begin{figure}[ht]
\begin{tabular}{c}
\centerline{\includegraphics[scale=0.6,trim=0 0 0 0]{gselaser.pdf}}
\end{tabular}
\caption{\small{The GSE semiconductor laser.}}
\label{phtd3Dview}
\end{figure}

\section{First Section Heading}

This section tests the display of section headings.  Also,
consideration should be given to the spacing before and after
paragraphs and headings.

\subsubsection{Maxwell's Equations}
The four equations can be presented in differential or integral
form. They are listed below in both forms.

\begin{equation}
\begin{array}{c c}
\nabla \times \mathbf{E} = \frac{-\partial{\mathbf{B}}}{\partial{t}} &
\qquad \oint \mathbf{E} \cdot dl = \frac{-\partial}{\partial{t}}\int
\mathbf{B} \cdot d \mathbf{S} \\
\nabla \times \mathbf{H} = \mathbf{J} + 
\frac{-\partial{\mathbf{D}}}{\partial{t}} &
\qquad \oint \mathbf{H} \cdot dl = \int \mathbf{J} \cdot d \mathbf{S}
+ \frac{\partial}{\partial{t}}\int_{area} \mathbf{D} \cdot d \mathbf{S} \\
\nabla \cdot \mathbf{B} = 0 &
\qquad \int \mathbf{B} \cdot d\mathbf{S} = 0 \\
\nabla \cdot \mathbf{D} = \rho &
\qquad \int \mathbf{D} \cdot d\mathbf{S} = Q_{enclosed}
\end{array}
\end{equation}

$\mathbf{E}$ and $\mathbf{H}$ are electric and magnetic field
amplitudes, whereas $\mathbf{D}$ and $\mathbf{B}$ are electric and
magnetic flux densities respectively. $\mathbf{E}$, $\mathbf{H}$,
$\mathbf{D}$ and $\mathbf{D}$ are continuous functions of space and
time with continuous derivatives.

The integral form of the curl equations are derived from the
differential forms by application of Stokes Theorem. This
theorem relates the curl o a vector function, A, into a line
integral of the function.

\begin{equation}
\int_{area} ( \nabla \times \mathbf{A} ) \cdot d \mathbf{S} = 
\oint_{loop} \mathbf{A} \cdot dl
\end{equation}

where $d \mathbf{S}$ and $d\mathbf{l}$ are unit vectors oriented normal
to the surface. For the divergence equations, Gauss' divergence theorem

Wave equation in terms of electric field amplitude reduced to its
homogenous form is given by

\begin{equation}
\nabla^2 \mathbf{E} - \mu \epsilon \frac{\partial^2 
  \mathbf{E}}{\partial t^2} = 0
\end{equation}

The wave equation in terms of magnetic field amplitude is 

\begin{equation}
\nabla^2 \mathbf{H} - \mu \epsilon \frac{\partial^2 
  \mathbf{H}}{\partial t^2} = 0
\end{equation}
