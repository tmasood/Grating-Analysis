
\chapter{INTRODUCTION} \label{ch:introduction}

\section{First Section Heading}

This section tests the display of section headings.  Also,
consideration should be given to the spacing before and after
paragraphs and headings.

\subsection{Maxwell's Equations}
The four equations can be presented in differential or integral
form. They are listed below in both forms.

\begin{equation}
\begin{array}{c c}
\nabla \times \mathbf{E} = \frac{-\partial{\mathbf{B}}}{\partial{t}} &
\qquad \oint \mathbf{E} \cdot dl = \frac{-\partial}{\partial{t}}\int
\mathbf{B} \cdot d \mathbf{S} \\
\end{array}
\label{curlE}
\end{equation}

\begin{equation}
\begin{array}{c c}
\nabla \times \mathbf{H} = \mathbf{J} + 
\frac{-\partial{\mathbf{D}}}{\partial{t}} &
\qquad \oint \mathbf{H} \cdot dl = \int \mathbf{J} \cdot d \mathbf{S}
+ \frac{\partial}{\partial{t}}\int_{area} \mathbf{D} \cdot d \mathbf{S}
\end{array}
\label{curlH}
\end{equation}

\begin{equation}
\begin{array}{c c}
\nabla \cdot \mathbf{B} = 0 &
\qquad \int \mathbf{B} \cdot d\mathbf{S} = 0 \\
\nabla \cdot \mathbf{D} = \rho &
\qquad \int \mathbf{D} \cdot d\mathbf{S} = Q_{enclosed}
\end{array}
\end{equation}

$\mathbf{E}$ and $\mathbf{H}$ are electric and magnetic field
amplitudes, whereas $\mathbf{D}$ and $\mathbf{B}$ are electric and
magnetic flux densities respectively. $\mathbf{E}$, $\mathbf{H}$,
$\mathbf{D}$ and $\mathbf{D}$ are continuous functions of space and
time with continuous derivatives.

Assuming simple linear relationship between electric vectors.

\begin{equation}
\mathbf{D} = \varepsilon \mathbf{E}
\label{Ddef}
\end{equation}

Similarly relationship between $\mathbf{H}$ and $\mathbf{B}$ is given by

\begin{equation}
\mathbf{B} = \mu \mathbf{H}
\label{magflux}
\end{equation}

The displacement vector D satisfies the equation below in the absense
of electric charges.

\begin{equation}
\mathbf{\nabla \cdot D} = 0
\label{defD0}
\end{equation}

The flow of electromagnetic power is space is described by the Poynting vector

\begin{equation}
\mathbf{S} = \mathbf{E} \times \mathbf{H}
\end{equation}

The power $\mathbf{P}$ flowing through an area A with an outward-directed normal unit vector n is given by

\begin{equation}
\mathbf{P} = \int_{\mathbf{A}} \mathbf{S} \cdot \mathbf{n}\, d{A}
\end{equation}

The time averaged poynting vector is given by

\begin{equation}
\mathbf{\bar{S}} = \frac{1}{2}[\mathbf{E} \times \mathbf{H^*}]
\end{equation}

\subsection{Wave Equation}

Substituting (\ref{magflux}) into (\ref{curlE}) and taking the curl on
both sides.

\begin{equation}
\nabla \times (\nabla \times \mathbf{E}) = 
-\mu \frac{\partial{}}{\partial{t}}(\mathbf{\nabla \times H})
\label{curlcurlE}
\end{equation}

Where $\mu$ is constant. Substituting (\ref{curlH}) and (\ref{Ddef})
into (\ref{curlcurlE}).

\begin{equation}
\nabla \times (\nabla \times \mathbf{E}) + \varepsilon \mu
\frac{\partial^2{E}}{\partial{t^2}}
\label{curlcurlE2}
\end{equation}

Equation (\ref{curlcurlE2}) holds even when $\varepsilon$ varies.

\begin{equation}
\nabla \times (\nabla \times \mathbf{E}) =
\nabla (\nabla \cdot \mathbf{E}) - \nabla^2 \mathbf{E}
\label{vident}
\end{equation}

Using relationship (\ref{vident}), we can write (\ref{curlcurlE2}) as

\begin{equation}
\nabla (\nabla \cdot \mathbf{E}) - \nabla^2 \mathbf{E} + \varepsilon \mu
\frac{\partial^2{E}}{\partial{t^2}} = 0
\label{divergradE}
\end{equation}

Using (\ref{Ddef}) and product rule (\ref{divproductrule})

\begin{equation}
\nabla \cdot \left(\frac{1}{\varepsilon}\mathbf{D}\right)=
 \left(\nabla \frac{1}{\varepsilon}\right)
\cdot D + \frac{1}{\varepsilon}(\nabla \cdot D)
\label{divproductrule}
\end{equation}

And in the absence of electric charges (\ref{defD0}), we can write equation
(\ref{divergradE}) as

\begin{equation}
\nabla^2 \mathbf{E} + \nabla \left[ \mathbf{E} \cdot \frac{\nabla \varepsilon}
{\varepsilon}\right]
 = \varepsilon \mu \frac{\partial^2 \mathbf{E}}{\partial t^2}
\label{earlywaveeq}
\end{equation}

When $\varepsilon$ is constant then

\begin{equation}
\nabla^2 \mathbf{E}
 = \varepsilon \mu \frac{\partial^2 \mathbf{E}}{\partial t^2}
\label{waveeq}
\end{equation}

The integral form of the curl equations are derived from the
differential forms by application of Stokes Theorem. This
theorem relates the curl o a vector function, A, into a line
integral of the function.

\begin{equation}
\int_{area} ( \nabla \times \mathbf{A} ) \cdot d \mathbf{S} = 
\oint_{loop} \mathbf{A} \cdot dl
\end{equation}

where $d \mathbf{S}$ and $d\mathbf{l}$ are unit vectors oriented normal
to the surface. For the divergence equations, Gauss' divergence theorem

Wave equation in terms of electric field amplitude reduced to its
homogenous form is given by

\begin{equation}
\nabla^2 \mathbf{E} - \mu \epsilon \frac{\partial^2 
  \mathbf{E}}{\partial t^2} = 0
\end{equation}

The wave equation in terms of magnetic field amplitude is 

\begin{equation}
\nabla^2 \mathbf{H} - \mu \epsilon \frac{\partial^2 
  \mathbf{H}}{\partial t^2} = 0
\end{equation}
