\documentclass{article}
\title{User's Guide to the Floquet program}
\author{Johannes Tausch}
\begin{document}
\maketitle

\section{Installation}
These instructions are for a UNIX system. It should be easy to get the
code running under WINDOWS, but then you are on your own.
Make sure you have LAPACK installed on your system, otherwise the code
won't compile. 

You also need the Bessel library. The library is a subset of the much
larger SLATEC library, which can be downloaded from www.netlib.org.
For convenience the the file \texttt{libbessel.tar.gz} is provided.
Put this file in a separate directory,
unzip and untar it using the commands\\
\leftline{$\quad$\texttt{gzip -d libbessel.tar.gz}}
\leftline{$\quad$\texttt{tar -xf libbessel.tar}} Next, type \texttt{make
  libbessel.a} to compile the library. There might be some warnings
about unrecognized options which are harmless. The result of the
compilation should be the file \texttt{libbessel.a} which must be
placed in a directory where the linker can find it.

Next put the file \texttt{tar -xf floquet.tar.gz} in a separate
directory, then unzip and untar it with the commands\\
\leftline{$\quad$\texttt{gzip -d floquet.tar.gz}}
\leftline{$\quad$\texttt{tar -xf floquet.tar}} Next, look at the
\texttt{Makefile}, and make the appropriate changes.  It may be
necessary to specify the path and name of the LAPACK and the bessel
libraries, the C compiler (must be gcc), the linker (must be g77) and
the name of the geometry description file that you want to link with
the code. In most cases, this should be pretty straight forward to do.
In the package are the geometry files of four example structures.
There are also shell scripts that show the command line options.

\section{Basic Usage via Command Line Options}
This is a list of command line options. It is important to put an
equal sign between the option and its value, e.g. \texttt{-gr=0.005}
and \emph{not} \texttt{-gr0.005}. If the parameter has a default
then this value will be used, unless the parameter is
specified in the command line. 

\subsection{The most important parameters}
\begin{itemize}
  \item[\texttt{-L}] Grating period.
  \item[\texttt{-W}] Width of the interior region (typically, the
  tooth height of the grating). 
  \item[\texttt{-d}] Length of the largest panel in the discretization
  of the interior problem. Decreasing this value will increase the
  numerical accuracy but slow things down at the same time 
  (default \texttt{-dx=0.05}).
  \item[\texttt{-p}] Number of space harmonics outside the grating region.
  (default \texttt{-p=3}). Increase this value if more accuracy is
  required, at the same time decrease the meshwidth with the
  \texttt{-d} option. 
  \item[\texttt{-gr}] Initial guess of the propagation constant, real
  part. Note that this is the attenuation factor (no default value).
  \item[\texttt{-gr}] Initial guess of the propagation constant, imaginary
  part (no default value).
  \item[\texttt{-k}] Frequency $\omega$. 
\end{itemize}
The grating period and the grating width are the dimensions of the
rectangle in the $xz$-plane that is the interior region. If these
parameters are fixed, they can also be defined in the geometry file.
In this case the command line options will be ignored.

\subsection{Parameters for controlling the output.}
Normally the code displays only the basic input characters and the
calculated propagation constant. If the verbosity level \texttt{-v} is
increased, then the iteration history and the transverse wave numbers
in the infinite regions are printed. 

There are two ways in which the mode is displayed. One is to give a
'cross section' of the mode for $z=0$ as a function of $x$. The data
is stored in the file \texttt{sol.<omega>.<beta>} in in the format 
x-value  Re(u(x,0))  Im(u(x,0)). The other way is to print the real
and imaginary parts of the mode $u(x,z)$ in the interior region on a
uniform mesh. This information is stored in the files 
\texttt{solRe.<omega>.<beta>} and \texttt{solIm.<omega>.<beta>}. Both
types of files can be loaded into Matlab and displayed graphically.
\begin{itemize}
  \item[\texttt{-v}] Level of verbosity; \texttt{-v=0} to suppress all
  output except for the final propagation constant. A nonzero value
  prints iteration history, transverse wave numbers in the infinite
  regions, etc (default \texttt{-v=0}).
  \item[\texttt{-a0}] Left endpoints of the $x$-interval in which a
  cross section of the mode for $z=0$ is generated. The data is stored
  in a \texttt{sol} file. The endpoint should be in an infinite layer.
  If the parameter is not given then the \texttt{sol}-file is not generated. 
  \item[\texttt{-a1}] Right endpoint, should be in an infinite layer. 
\item[\texttt{-N}] Number of points in a \texttt{sol}-file at which
  the solution is printed.   (default \texttt{-N=200}).
\item[\texttt{-Nx}] Number of points in a \texttt{solRe/Im}-file in
  $x$-direction.
\item[\texttt{-Nz}] Number of points in a \texttt{solRe/Im}-file in
  $z$-direction. If both \texttt{Nx} and \texttt{Nz} are not given
  then no output is generated. If one of the two is missing, it will
  be set equal to the given one.
\end{itemize}

\subsection{Parameters that normally do not need to be specified.}

This is a list of parameters where the default value usually gives the
desired results.
\begin{itemize}
  \item[\texttt{-q}] Order of the quadrature rule used for the
  computation of the BEM matrix in the interior problem. Must be
  between 2 and 12 (default \texttt{-q=3}).
  \item[\texttt{-M}] Maximal number of iterations of nonlinear solver (default
  \texttt{-M=20}).
  \item[\texttt{-E}] Tolerance of the nonlinear solver (default
  \texttt{-E=1e-9}).
\end{itemize}


\section{How to Write Your Own geom File} 
The best way to learn it is to look at the provided geometry files;
however, some proficiency in C-programming is required. The best
starting point is the file \texttt{geomSimple.c} which describes a
simple guide with a core, and a superstrate and substrate, as shown in
the figure in the comment section of the file.

Every geometry file contains the subroutine \texttt{initExterior()}
which describes the uniform layers and the the subroutine
\texttt{initInterior()} which describes the interior (or grating)
region. Finally, it contains the routine \texttt{domainNr()} which is
only necessary if plots of the interior solution are to be generated.

\subsection{Exterior Regions.} The routine \texttt{initExterior()}
initializes the following global variables 
\begin{itemize}
\item[] \texttt{nTrnsRight} number of
translation operators on the right of the grating region. Every finite
uniform layer has a translation operator.
\item[] \texttt{nTrnsLeft}  number of
translation operators on the left of the grating region.
\item[] \texttt{kapRight[], kapLeft[]} The wave numbers of the
  layers. Note that we start counting with 1 beginning from the
  grating region towards the infinite region.  \texttt{kapRight[0]} and
  \texttt{kapLeft[0]} are the wave numbers of the infinite regions.
\item[] \texttt{wRight[], wLeft[]} are the widths of the 
  layers. The numbering scheme is the same as for the wave numbers. 
  \texttt{wRight[0]} and \texttt{wLeft[0]} are not used.
\end{itemize} 
It is important to allocate all arrays. 


\subsection{Interior Region} 
The geometry description happens in the
routine \texttt{initInterior()}. More specifically, this routine sets up
a linked list of \texttt{domain}-structures and a linked list of
\texttt{pnlLst}-structures.  
The interior domain is a rectangle in the $xz$-plane. The lower left
corner is the origin $(0,0)$ the upper right corner is given by the
variables (\texttt{widthGrating},\texttt{gratingPeriod}). These
variables are either initialized with the \texttt{-W} and \texttt{-L}
command line options or initialized in \texttt{initInterior()}, in
which case the command line options are ignored. 

The interior region is divided into domains with constant index.
There is one \texttt{domain}-structure for every domain The domains are
bounded by straight panels, but can have arbitrary shape otherwise.
Also, the number of domains is arbitrary, this allows for gratings
that are etched through several layers. The following members of the
\texttt{domain}-structure must be initialized in
\texttt{initInterior()}.
\begin{itemize}
\item[]\texttt{kappa} wave number (=index times frequency)
\item[]\texttt{nPnls} number of panels that describe the boundary of
  the domain. 
\item[]\texttt{pnls[]} Array of the indices of boundary panels. The
  indices are counted from zero and refer to the position in the
  linked list of panels described below. If the normal of a type-1
  panel (explanation below!) points into the domain then the index
  gets a 'minus' sign. Note that the array must be
  allocated before it is written.
\item[]\texttt{next} Pointer to the next domain in the linked list. If
  the pointer is \texttt{NULL} then this is interpreted as the end of
  the list of domains.
\end{itemize}

There is one \texttt{pnlLst}-structure for every panel. The members
that must be initialized in \texttt{initInterior()} follow
\begin{itemize}
\item[]\texttt{type} The type of the panel. There are five types:
\begin{itemize}
  \item[]\texttt{type=0} Panel is on the boundary $z=0$ of the rectangle.
  \item[]\texttt{type=1} Panel is in the interior of the rectangle.
  \item[]\texttt{type=2} Panel is on the boundary \texttt{z=gratingPeriod}.
  \item[]\texttt{type=3} Panel is in the boundary $x=0$.
  \item[]\texttt{type=4} Panel is in the boundary \texttt{x=widthGrating}.
\end{itemize}
\item[]\texttt{x0[]} $x$- and $z$- coordinates of the first end point.
\item[]\texttt{x1[]} $x$- and $z$- coordinates of the second end
  point. The normal points into the left of the panel as we go from
  the first to the second end point.
\item[]\texttt{next} Pointer to the next panel in the linked list. If
  the pointer is \texttt{NULL} then this is interpreted as the end of
  the list.
\end{itemize}
It is important that the type-0 and type-2 panels are in one-to-one
correspondence, i.e., the discretization of the lower and upper side
of the rectangle should be identical. Also, keep in
mind that the panels described in \texttt{initInterior()} are not the
panels used for the discretization. Instead, the program refines the
panels automatically such that every refined panel is no longer than
the value given by the command line option \texttt{-d}. Therefore it
is better to describe the geometry with as few panels as possible.

\subsection{A Simple Example}
To illustrate the geometry description files consider the waveguide
shown in Figure~\ref{fig-domain} which is implemented by
\texttt{geomSimple.c}. 

\begin{figure}[htb]
\begin{center}
\setlength{\unitlength}{2cm}
\begin{picture}(5,3)
\put(5.1,0.5){\vector(1,0){0.5}}   % arrows for directions
\put(5.22,0.25){$x$}
\put(3,2.7){\vector(0,1){0.25}}
\put(3.15,2.88){$z$}
\put(0,0.5){\line(1,0){5}}         % horizontal lines
\put(0,1.5){\line(1,0){5}}
\put(0,2.5){\line(1,0){5}}
\thicklines
\put(2,0.4){\line(0,1){2.2}}
\put(3.5,0.4){\line(0,1){0.1}}    % grating layer boundaries
\put(3,1){\line(0,1){0.5}}
\put(3.5,0.5){\line(0,1){0.5}}
\put(3,2){\line(0,1){0.5}}
\put(3.5,1.5){\line(0,1){0.5}}
\put(3,2.5){\line(0,1){0.1}}
\put(3,0.5){\line(1,0){0.5}}
\put(3,1){\line(1,0){0.5}}
\put(3,1.5){\line(1,0){0.5}}
\put(3,2){\line(1,0){0.5}}
%\put(3,2.5){\line(1,0){0.5}}
\put(1.4,0.95){$\Omega_1$}
\put(2.5,0.95){$\Omega_2$}
\put(4.1,0.95){$\Omega_3$}
\put(2.95,0.25){$0$}
\put(3.45,0.25){$w$}
\put(1.5,0.1){$\mathbf{\vdots}$}  % continuation signs
\put(1.5,2.8){$\mathbf{\vdots}$}
\end{picture}
\end{center}
\caption{The guide described in \texttt{geomSimple.c}}
\label{fig-domain}
\end{figure}

This example consists of the core with grating, a sub- and a
superstrate. The interior region is the rectangle
$[0,w]\times[0,\Lambda]$ in the $xz$-plane.  

Let's first have a closer look at the routine \texttt{initExterior()}.
The right exterior region is air, therefore there is no translation
operator in the right region and \texttt{nTrnsRight=0}. The left side
has one finite layer, which amounts to one translation operator,
\texttt{nTrnsLeft=1}. The width and wavenumber are specified as
\texttt{wLeft[1] = 0.6366198} and \texttt{kapLeft[1] = omega*K2}. The
wavenumber in the infinite layers are specified as \texttt{kapLeft[0]
  = omega*K1} and \texttt{kapRight[0] = omega*K3}. The widths
\texttt{wLeft[0] = wRight[0] = 0.0} are never used and therefore set
to zero. Note how the arrays are allocated.

\begin{figure}[htb]
\begin{center}
{% Picture saved by xtexcad 2.4
\unitlength=0.6000pt
\begin{picture}(230.00,370.00)(0.00,0.00)
\thicklines
\put(210.00,350.00){\vector(-1,0){100.00}}
\put(20.00,190.00){\vector(0,1){70.00}}
\put(210.00,190.00){\vector(0,1){80.00}}
\put(210.00,190.00){\vector(-1,0){100.00}}
\put(20.00,20.00){\vector(0,1){90.00}}
\put(210.00,20.00){\vector(0,1){90.00}}
\put(20.00,20.00){\vector(1,0){100.00}}
\put(210.00,190.00){\line(-1,0){190.00}}
\put(210.00,350.00){\line(-1,0){190.00}}
\put(210.00,20.00){\line(0,1){330.00}}
\put(20.00,20.00){\line(1,0){190.00}}
\put(20.00,350.00){\line(0,-1){330.00}}
\put(110,190.00){\vector(0,-1){70.00}}
\put(115,150.00){$n$}
\put(0,102){$P_3$}
\put(0,267){$P_4$}
\put(220,102){$P_5$}
\put(220,267){$P_6$}
\put(110,0){$P_0$}
\put(110,195){$P_2$}
\put(110,355){$P_1$}
\put(110,267){$\Omega_3$}
\put(110,102){$\Omega_2$}
\put(0,0){$(0,0)$}
\put(215,355){$(w,\Lambda)$}
\end{picture}}
\end{center}
\caption{The interior region of \texttt{geomSimple.c}}
\label{fig-inner}
\end{figure}

The interior region is shown in Figure~\ref{fig-inner}. It consists of two
domains $\Omega_2$, $\Omega_3$ which are bounded by seven panels
$P_0,\dots, P_6$. The upper domain is bounded by the panels $P_2, P_4,
P_1, P_6$, the lower one by $P_0,P_5,P_2,P_3$. Panel $P_0$ is type-0,
panel $P_1$ is type-2, panel $P_2$ is type-1, panels $P_4$ and $P_5$
are type-3, panels $P_6$ and $P_7$ are type-4.

The orientation of the panels is indicated in the by figure by the
arrows. The normal points into the left side of the panel, if the
panel is traversed in its orientation, hence the normal of panel $P_2$
points into the domain $\Omega_2$. This is why $P_2$ appears as '-2'
in $\Omega_2$'s panel list. The orientation is only of concern for
type-1 panels, therefore the other panels are listed as 'positive',
even if their normal points into the domain.

\end{document}
%%% Local Variables: 
%%% mode: latex
%%% TeX-master: t
%%% End: 
